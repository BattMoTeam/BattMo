\documentclass{standalone}

\usepackage{caladea,carlito}
\usepackage{amsmath}
\usepackage{colortbl}
\usepackage{listings}
\usepackage{color}
\usepackage{textcomp}
\definecolor{listinggray}{gray}{0.9}
\definecolor{lbcolor}{rgb}{0.95,0.95,0.95}
\definecolor{lightgray}{gray}{0.3}
\lstset{%
  xleftmargin= 4pt,
  xrightmargin=4pt}
\lstdefinelanguage{MRST}{%
  alsoletter={...},%
  morekeywords={%                             % keywords
  break,case,catch,continue,elseif,else,end,for,function,global,classdef,properties,%
  if,otherwise,persistent,return,switch,try,while,...},%
  comment=[l]\%,%                             % comments
  morecomment=[l]...,%                        % comments
  morestring=[m]',%                           % strings
}[keywords,comments,strings]%
\lstset{
  literate={~}{$\sim$}{1},
  literate={-}{\textendash}{1},
  backgroundcolor=\color{lbcolor},
  tabsize=4,
  rulecolor=,
  language=MRST,
  basicstyle=\small,
  moredelim=**[is][\color{red}]{@@}{@@},
  ,moredelim=**[is][\color{orange}]{££}{££},
  upquote=true,
  aboveskip={0.5\baselineskip},
  columns=fixed,
  showstringspaces=false,
  extendedchars=true,
%  breaklines=true,
  prebreak = \raisebox{0ex}[0ex][0ex]{\ensuremath{\hookleftarrow}},
  frame=single,
  showtabs=false,
  showspaces=false,
  identifierstyle=\ttfamily,
  keywordstyle=\color[rgb]{0,0,1},
  commentstyle=\color[rgb]{0.133,0.545,0.133},
  stringstyle=\color[rgb]{0.627,0.126,0.941},
}
\newcommand{\mcode}[1]{\lstinline|#1|}

\usepackage{tikz}
\usetikzlibrary{arrows, positioning, calc, patterns, backgrounds, arrows.meta, matrix}
\usetikzlibrary{graphs, graphdrawing, graphdrawing.layered}
\usetikzlibrary{shapes.multipart}
\usegdlibrary{trees}

%% Improved boxes
\usepackage[most]{tcolorbox}
\tcbset{colback=red!5!white,colframe=red!75!black,left=3pt,right=3pt,top=3pt,bottom=3pt}

\usetikzlibrary{shapes.multipart}
\usetikzlibrary{3d}

\newcommand{\fracpar}[2]{\frac{\partial #1}{\partial #2}}
\newcommand{\norm}[1]{\left\|#1\right\|}
\newcommand{\tens}[1]{\ensuremath{\boldsymbol{\mathsf{#1}}}}
\newcommand{\vect}[1]{\ensuremath{\boldsymbol{#1}}}
\newcommand{\mat} [1]{\ensuremath{\boldsymbol{#1}}}
\newcommand{\ddiv }  {\ensuremath{\mathtt{div}}}%
\newcommand{\dgrad}  {\ensuremath{\mathtt{grad}}}%
\newcommand{\davg}   {\ensuremath{\mathtt{avg}}}%
\newcommand{\vv}     {\ensuremath{\vect{v}}}
\newcommand{\vp}     {\ensuremath{\vect{p}}}
\newcommand{\vq}     {\ensuremath{\vect{q}}}
\newcommand{\ad}[1]  {\ensuremath{\left\langle #1 \right\rangle}}
\newcommand{\trans}{T}
\newcommand{\dt}{\partial_{t}}
\newcommand{\abs}[1]{\left| #1\right|}
\newcommand{\grad}{\nabla}
\newcommand{\dive}{\grad\cdot }
\newcommand{\Real}{\mathbb{R}}
\newcommand{\brac}[1]{\left<#1\right>}
\newcommand{\trace}{\mathrm{tr}}
\newcommand{\sym}{\ensuremath{\text{sym}}}

\begin{document}

\def\mynode #1#2{
  \begin{tcolorbox}[width = #1cm]
    \begin{center}
      #2
    \end{center}
  \end{tcolorbox}
}
\def\myobjnode #1#2{
  \begin{tcolorbox}[width = #1cm,colback=white,colframe=green!75!black ]
    \begin{center}
      #2
    \end{center}
  \end{tcolorbox}
}

\def\myscale{0.7}
\def\mcode #1{\lstinline[basicstyle={\setmainfont{Freemono}[Scale = \myscale, FakeBold = 1.5]\setsansfont{Freemono}[Scale = \myscale, FakeBold = 1.5]\setmonofont{Freemono}[Scale = \myscale, FakeBold = 1.5]}]|#1|}
\def\mcodeclass #1{\lstinline[basicstyle={\setmainfont{Freemono}[Scale = \myscale, FakeBold = 1.5]\setsansfont{Freemono}[Scale = \myscale, FakeBold = 1.5]\setmonofont{Freemono}[Scale = \myscale, FakeBold = 1.5]\color{blue}}]|#1|}

\newdimen\mydim
\def\myobjnode #1#2{\mydim=#1cm\begin{tcolorbox}[leftright skip = 0pt, width=\myscaling\mydim, colback=white,colframe=green!75!black, boxsep=0pt,left=0pt,right=0pt,top=2mm,bottom=2mm]
    \begin{center}
      #2
    \end{center}
  \end{tcolorbox}}
\def\myscaling{1.2}

\def\nudgenumb {1cm}

\def\baselinestretch{0.8}
\setmainfont{Freemono}[Scale = \myscale]\setsansfont{Freemono}[Scale = \myscale]\setmonofont{Freemono}[Scale = \myscale]
\begin{tikzpicture}
  \graph [tree layout, sibling sep = 2mm, level distance = 1.5cm, edges={thick, -{Latex[length=2mm, width=2mm]}}, edge=rounded corners, nodes = {outer sep = 0mm, inner sep = 0mm}, ]
  { battery [as = \myobjnode{2}{\mcodeclass{Battery}}] ->
    { control [as = \myobjnode{2}{\mcode{Control}\\ (\mcodeclass{ControlModel})}, target edge style = {bend right = 10pt}] ,
      poselde [as = \myobjnode{2.3}{\mcode{PositiveElectrode}\\ (\mcodeclass{Electrode})}] ,
      elyte   [as = \myobjnode{2}{\mcode{Electrolyte}\\ (\mcodeclass{Electrolyte})}] ,
      negelde [as = \myobjnode{2.3}{\mcode{NegativeElectrode}\\ (\mcodeclass{Electrode})}] ,
      thermal [as = \myobjnode{1.8}{\mcode{Thermal}\\ (\mcodeclass{ThermalModel})}, target edge style = {bend left = 10pt}]
    }
  }; 
\end{tikzpicture}

\end{document}

% Local Variables:
% TeX-engine: luatex
% End: